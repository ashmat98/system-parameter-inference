\begin{frame}{Measurement of the Amplitude}
\begin{align*}
\onslide<1->{&\theta = \theta_0 \, e^{-\gamma t} \cos(\omega_1 t) \\}
\onslide<2->{\Rightarrow \; &a_n = a_0 \, e^{-n \gamma T} \text{  were  } T=\cfrac{2 \pi}{\omega_1} \\}
\onslide<3>{\Rightarrow \; &\ln(a_n) = ln(a_0) - n \gamma T }
\end{align*}
\end{frame}


\begin{frame}{Measurement of the Amplitude}
\framesubtitle{results}
\begin{figure}[ht]
	\centering	
	\begin{tikzpicture}
	\begin{axis}[
	legend pos=south east,
	width=0.9\linewidth,
	height=6cm,
	title={$ln(\text{amplitude})$ plotted as a function of the number of oscillations},
	xlabel={Number of oscillations, n},
	ylabel={$ln(\text{amplitude a\textsubscript{n} / arbitrary units})$ },
	xmajorgrids=true,ymajorgrids=true,
	grid style=dashed,
	]
	
	\addplot[
	color=blue,	only marks, mark=+,
	mark options={solid,scale=2,line width=0.5pt},
	error bars/.cd,
	y dir=both, y explicit,
	error bar style={line width=0.5pt,solid},
	error mark options={line width=0.7pt,mark size=3pt,rotate=90}
	]		
	table[x index = {0}, y index = {3}, y error plus index=4, y error minus index=5, col sep=comma]
	{data/transient_response_0.0A.csv};
	\addlegendentry{braking current = \SI{0.0}{\ampere}}
	
	\addplot [color=black,forget plot]
	table [col sep=comma, x index=0, 
	y={create col/linear regression={y=[index]3}}] 
	{data/transient_response_0.0A.csv};
	\xdef\slopeOne{\pgfplotstableregressiona}
	\xdef\interceptOne{\pgfplotstableregressionb}
	\node[] at (axis cs: 10,1.47){%
		$\pgfmathprintnumber{\slopeOne} \cdot x
		\pgfmathprintnumber[print sign]{\interceptOne}$};
	
	\addplot[color=orange, only marks, mark=+,
	mark options={solid,scale=2,line width=0.5pt},
	error bars/.cd,
	y dir=both, y explicit,
	error bar style={line width=0.5pt,solid},
	error mark options={line width=0.7pt,mark size=3pt,rotate=90}
	]		
	table[x index = {0}, y index = {3}, y error plus index=4, y error minus index=5, col sep=comma]
	{data/transient_response_0.6A.csv};
	\addlegendentry{braking current = \SI{0.6}{\ampere}}
	
	\addplot [color=black]
	table [col sep=comma, x index=0, 
	y={create col/linear regression={y=[index]3}}] 
	{data/transient_response_0.6A.csv};
	\xdef\slopeTwo{\pgfplotstableregressiona}
	\xdef\interceptTwo{\pgfplotstableregressionb}
	\node[] at (axis cs: 6.7,-0.0){%
		$\pgfmathprintnumber[precision=3]{\slopeTwo} \cdot x
		\pgfmathprintnumber[print sign]{\interceptTwo}$};
	
	\end{axis}
	\end{tikzpicture}
	\caption{$\ln(\text{amplitude})$ versus oscillation number $n$. Dependence is linear as we expected.}
	\label{fig:log-plot}
	\end{figure}

\end{frame}

\begin{frame}{Results}
\begin{align*}
\ln(a_n) = ln(a_0) - n \gamma T \\
\text{slope of the line} =  -\gamma \, T = -\gamma \cfrac{2 \pi}{\omega_1}
\end{align*}
\pause

For low damping $\omega_1 \approx \omega_0$ so the slope is approximately $-\cfrac{\pi}{Q}$.
\begin{table}\centering
\begin{tabular}{c c} 
	\toprule
	slope [$I_b=0.0A]$ & \num{-0.044 \pm 0.001}\\
	slope [$I_b=0.6A]$ & \num{-0.80 \pm 0.03} \\
	\bottomrule
\end{tabular}
\end{table}
\pause
And by assumption $\text{slope} \approx -\cfrac{\pi}{Q}$
\begin{table}[H]\centering
	\begin{tabular}{c c} 
		\toprule
		Q [$I_b=0.0A$] & \num{71 \pm 1} \\
		Q [$I_b=0.6A$] & \num{3.9 \pm 0.1} \\
		\bottomrule
	\end{tabular}
\end{table}

\end{frame}

